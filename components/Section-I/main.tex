
\newpage



\section{\textbf{LỜI MỞ ĐẦU}}
Trong thời đại thương mại điện tử phát triển mạnh mẽ, việc đánh giá và phân tích cảm xúc (sentiment analysis) của khách hàng đối với sản phẩm đã trở thành yếu tố quan trọng giúp doanh nghiệp cải thiện chất lượng dịch vụ và người dùng đưa ra quyết định mua hàng sáng suốt. Các nền tảng thương mại điện tử hiện đại không chỉ cung cấp không gian mua sắm mà còn cần tích hợp trí tuệ nhân tạo (artificial intelligence) để phân tích phản hồi của khách hàng một cách tự động và chính xác.

Hệ thống đánh giá sản phẩm với AI ra đời nhằm giải quyết bài toán phân tích cảm xúc từ bình luận và đánh giá của khách hàng, sử dụng công nghệ xử lý ngôn ngữ tự nhiên (natural language processing) tiên tiến. Với khả năng phân tích tự động các bình luận thành các mức độ hài lòng (tích cực/positive, tiêu cực/negative, trung tính/neutral), hệ thống giúp doanh nghiệp thấu hiểu nhanh chóng phản hồi của khách hàng, từ đó cải thiện chất lượng sản phẩm và dịch vụ.

Chính vì vậy, nhóm quyết định thực hiện đồ án “Xây dựng hệ thống đánh giá sản phẩm với AI” với mục tiêu phát triển một hệ thống hoàn chỉnh bao gồm Web App và Mobile App, tích hợp mô hình AI sử dụng PhoBERT để phân tích cảm xúc tiếng Việt, cho phép người dùng xem sản phẩm, đánh giá và nhận được phân tích tự động về mức độ hài lòng của cộng đồng.