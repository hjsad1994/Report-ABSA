
\newpage

\section{\textbf{TỔNG QUAN VỀ VERITASHOP}}

\subsection{Giới thiệu}
Trong bối cảnh TMĐT bùng nổ, nhu cầu tự động phân tích cảm xúc và dự đoán mức độ hài lòng từ bình luận sản phẩm ngày càng cấp thiết. VeritaShop tập trung phát triển mô hình NLP (PhoBERT) để phân loại cảm xúc tiếng Việt chính xác và tích hợp vào nền tảng TMĐT kiến trúc microservices, nhằm nâng cao trải nghiệm người dùng và hỗ trợ quyết định dựa trên dữ liệu.

\subsection{Tổng quan hệ thống}
\subsubsection{Mục đích hệ thống}
VeritaShop là nền tảng TMĐT thông minh: hiển thị sản phẩm, thu nhận đánh giá và cung cấp phân tích cảm xúc tự động. Mục tiêu chính:
\begin{itemize}
    \item Nền tảng giao diện thân thiện, responsive, hỗ trợ sáng/tối.
    \item Tích hợp PhoBERT phân tích cảm xúc bình luận.
    \item Dự đoán mức độ hài lòng (tích cực/trung tính/tiêu cực) tự động.
    \item Tối ưu hiệu năng với microservices, containerization và cloud.
\end{itemize}
VeritaShop không cạnh tranh trực tiếp với các nền tảng lớn, mà cung cấp giải pháp quy mô vừa, ứng dụng AI để cải thiện phân tích cảm xúc, dự đoán xu hướng và trải nghiệm người dùng.

\subsubsection{Khảo sát các sản phẩm tương tự}
Các nền tảng như Shopee, Lazada, Tiki còn hạn chế về phân tích cảm xúc và dự đoán hài lòng:
\begin{itemize}
    \item \textbf{Shopee}: Đánh giá cơ bản, thiếu phân tích cảm xúc tự động và dự đoán xu hướng.
    \item \textbf{Lazada}: Giao diện tốt, chưa tích hợp AI phân tích cảm xúc chính xác.
    \item \textbf{Tiki}: Hỗ trợ media, còn yếu ở phân tích ngữ nghĩa/ngữ cảnh bình luận.
\end{itemize}
Hệ thống đề xuất cải tiến:
\begin{itemize}
    \item Tích hợp PhoBERT phân tích cảm xúc tiếng Việt.
    \item Dự đoán mức độ hài lòng tự động, chính xác.
    \item Trực quan hóa kết quả để hỗ trợ quyết định.
\end{itemize}

\subsubsection{Yêu cầu hoạt động của ứng dụng}
\subsubsubsection{Phần dành cho người dùng cuối}
Chức năng chính:
\begin{itemize}
    \item \textbf{Đăng ký/đăng nhập}: Xác thực email, JWT.
    \item \textbf{Hồ sơ}: Cập nhật thông tin, ảnh đại diện.
    \item \textbf{Duyệt/tìm kiếm}: Danh sách, bộ lọc, chi tiết sản phẩm.
    \item \textbf{Đánh giá/bình luận}: Viết đánh giá, xem phân tích cảm xúc cộng đồng.
    \item \textbf{Mua hàng}: Giỏ hàng, thanh toán, theo dõi đơn.
    \item \textbf{Hỗ trợ AI}: Chatbot gợi ý và trợ giúp.
\end{itemize}

\subsubsubsection{Phần dành cho người bán/doanh nghiệp}
\begin{itemize}
    \item \textbf{Sản phẩm}: CRUD danh mục/sản phẩm.
    \item \textbf{Thống kê}: Doanh số, đánh giá, mức độ hài lòng.
    \item \textbf{Phân tích cảm xúc}: Báo cáo chi tiết theo sản phẩm.
    \item \textbf{Đơn hàng}: Xử lý và theo dõi trạng thái.
    \item \textbf{Tương tác khách hàng}: Phản hồi bình luận/đánh giá.
\end{itemize}

\subsubsubsection{Phần dành cho quản trị viên}
\begin{itemize}
    \item \textbf{Người dùng}: Quản lý và khóa tài khoản.
    \item \textbf{Sản phẩm}: Duyệt và kiểm soát chất lượng.
    \item \textbf{Monitoring}: Theo dõi hiệu suất và sự cố.
    \item \textbf{Dashboard}: Quản trị tổng quan.
\end{itemize}

\subsection{Thiết kế tương tác}
Giao diện thân thiện, responsive, hỗ trợ sáng/tối trên desktop/mobile. Thành phần chính:
\begin{itemize}
    \item \textbf{Trang chủ}: Sản phẩm nổi bật/khuyến mãi, chỉ số hài lòng, lazy loading.
    \item \textbf{Trang sản phẩm}: Chi tiết, đánh giá và phân tích cảm xúc.
    \item \textbf{Giỏ hàng/Thanh toán}: Thao tác đơn giản, an toàn.
    \item \textbf{Hồ sơ}: Thông tin cá nhân, lịch sử mua, đánh giá.
    \item \textbf{Bảng điều khiển người bán}: Quản lý sản phẩm, đơn và thống kê.
    \item \textbf{Admin dashboard}: Quản lý hệ thống, người dùng, sản phẩm.
    \item \textbf{Chatbot AI}: Tìm kiếm, giải đáp, gợi ý.
\end{itemize}

\subsection{Phương pháp tiếp cận và giải quyết vấn đề}
\subsubsection{Mô hình tổng quát hệ thống}
Kiến trúc Microservices kết hợp công nghệ NLP và hạ tầng hiện đại:
\begin{itemize}
    \item \textbf{Frontend}: Next.js (Web), Flutter (Mobile) responsive, hiệu năng cao.
    \item \textbf{Microservices Backend}: Node.js (NestJS/TypeScript) cho nghiệp vụ; Python (FastAPI) cho AI.
    \item \textbf{AI Service}: PhoBERT phân tích cảm xúc bình luận.
    \item \textbf{Cơ sở dữ liệu}: PostgreSQL (cấu trúc), MongoDB (phân tích/logs).
    \item \textbf{DevOps}: Docker, Kubernetes, GitHub Actions, AWS.
\end{itemize}

\subsubsection{Phương pháp xây dựng phần mềm}
Áp dụng Agile trên kiến trúc Microservices, gồm:
\begin{itemize}
    \item Phân tích yêu cầu.
    \item Thiết kế kiến trúc (Microservices, Layered), CSDL và mô hình AI.
    \item Phát triển services (User/Auth, Product, Payment, AI) với Prisma ORM/PhoBERT.
    \item Tích hợp/kiểm thử; áp dụng Saga cho giao dịch phân tán.
    \item DevOps: Docker, Kubernetes, GitHub Actions CI/CD.
\end{itemize}

\subsubsection{Kiến trúc phần mềm}
Kiến trúc Microservices áp dụng Layered Architecture cho từng service:

Presentation Layer (Controllers):
- Xử lý HTTP, validation/authentication, logging

Business Logic Layer (Services):
- Logic nghiệp vụ, xử lý use cases, quản lý giao dịch

Data Access Layer (Repositories):
- Truy cập database, CRUD, tối ưu truy vấn, caching

\begin{itemize}
    \item \textbf{Frontend}: Next.js với App Router và Server Components cho Web App, Flutter với BLoC pattern cho Mobile App.
    \item \textbf{Microservices Backend}:
    \begin{itemize}
        \item Node.js với Express.js và TypeScript cho User/Auth, Product, Payment services được tổ chức theo Layered Architecture (Controllers → Services → Repositories)
        \item Python với FastAPI cho AI service phân tích cảm xúc với kiến trúc phân lớp rõ ràng
        \item Prisma ORM để truy cập cơ sở dữ liệu type-safe
    \end{itemize}
    \item \textbf{Cơ sở dữ liệu}: PostgreSQL cho dữ liệu có cấu trúc, MongoDB cho dữ liệu phân tích và logs.
    \item \textbf{Giao tiếp}: REST API và MQ (RabbitMQ/Kafka) giữa các service.
\end{itemize}

\subsubsection{Công nghệ triển khai hệ thống}
\subsubsubsection{Microservices Backend}
\begin{itemize}
    \item \textbf{Node.js (Express.js + TypeScript)}: Tổ chức theo Layered Architecture (Controllers/Services/Repositories) với middleware linh hoạt.
    \item \textbf{Python (FastAPI)}: Dịch vụ AI async hiệu năng cao, tự động hoá OpenAPI.
    \item \textbf{Prisma ORM}: Truy cập CSDL type-safe, hỗ trợ migrations và query builder.
    \item \textbf{PostgreSQL}: CSDL quan hệ cho dữ liệu nghiệp vụ (users, products, orders).
    \item \textbf{MongoDB}: NoSQL cho phân tích, logs và cache.
    \item \textbf{RabbitMQ/Kafka}: Hàng đợi thông điệp cho giao tiếp bất đồng bộ giữa services.
\end{itemize}

\subsubsubsection{Frontend}
\begin{itemize}
    \item \textbf{Next.js}: SSR/SSG với App Router và Server Components để tối ưu hiệu năng.
    \item \textbf{Flutter}: Ứng dụng di động đa nền tảng (iOS/Android) bằng Dart.
    \item \textbf{Redux Toolkit}: Quản lý state cho React.
    \item \textbf{Material-UI/Ant Design}: Thư viện UI thống nhất giao diện.
\end{itemize}

\subsubsubsection{AI/ML}
\begin{itemize}
    \item \textbf{PhoBERT}: Mô hình tiếng Việt SOTA cho phân tích cảm xúc.
    \item \textbf{Transformers (Hugging Face)}: Hạ tầng mô hình hoá và suy luận NLP.
    \item \textbf{TensorFlow/PyTorch}: Huấn luyện và triển khai mô hình học sâu.
    \item \textbf{Scikit-learn}: Tiền xử lý và đánh giá mô hình.
\end{itemize}

\subsubsubsection{DevOps \& Infrastructure}
\begin{itemize}
    \item \textbf{Docker}: Đóng gói ứng dụng và dependencies dưới dạng container.
    \item \textbf{Kubernetes}: Orchestrate và quản lý containers ở môi trường production.
    \item \textbf{GitHub Actions}: CI/CD tự động cho test/build/deploy.
    \item \textbf{AWS}: Hạ tầng cloud (EKS, RDS, S3, ECR, ...).
    \item \textbf{Prometheus \& Grafana}: Giám sát và cảnh báo.
    \item \textbf{ELK Stack}: Ghi log tập trung (Elasticsearch, Logstash, Kibana).
\end{itemize}

% \subsection{Tổng kết chương}
% Chương này đã trình bày tổng quan về VeritaShop - hệ thống phân tích cảm xúc sản phẩm sử dụng AI tiên tiến. Nội dung bao gồm giới thiệu dự án, khảo sát thị trường, yêu cầu chức năng, thiết kế giao diện và phương pháp triển khai. Hệ thống áp dụng kiến trúc microservices với mô hình phân lớp, sử dụng các công nghệ hiện đại để phân tích bình luận và dự đoán mức độ hài lòng khách hàng. Chương tiếp theo sẽ trình bày cơ sở lý thuyết về xử lý ngôn ngữ tự nhiên và mô hình học máy.
